\section{Port knocking prevention}
The idea behind the implemented port knocking prevention is to only allow a 
certain
IP address to connect to a service's port and conceal its existence to other 
potential clients such as port scanners. 
This means that any incoming SYN packets sent by a forbidden client will be 
answered with a RST packet. To achieve this,
two methods come to mind:
\begin{enumerate}
 \item Check all outgoing TCP packets' source port and destination address, set 
RST flag accordingly
 \item Intercept all socket lookups and check if port and address are allowed, 
fake negative results if not.
\end{enumerate}
To implement method 1, it is sufficient to hijack the 
\verb+tcp_transmit_skb()+ function in ~\verb+net/ipv4/tcp_output.c+ and alter 
the TCP flags found in the \verb+struct sk_buff+
parameter to RST. Although this method successfully prevents any undesired 
connections, tests showed that the kernel sends a SYN/ACK
packet to the client nevertheless. This causes a ``connection closed 
unexpectedly'' error or similar on the client side rather than a
``connection refused''. Furthermore, the server's internal TCP state machine is 
confused by this and retransmits the manipulated packet
multiple times. Therefore, this approach is clearly unsuitable for a rootkit. 
\par
This is why method 2 was chosen for the rootkit. It intervenes with the 
connection handling process directly when a packet arrives at transport layer.
In the Linux kernel, every IPv4 TCP packet is handled by the 
\texttt{tcp\_v4\_rcv()} function. This 
function uses the \texttt{\_\_inet\_lookup\_skb()} function to resolve the 
socket 
corresponding to the port number in the packet header. The rootkit hijacks this 
function call and
redirects it to our own method that checks IP and port number against the values 
provided in the rootkit's parameters. If the client IP
is not allowed, it simply returns a zero value or else calls the original 
lookup 
function.
However, method still is far from perfection since it can still be circumvented 
by spoofing an accepted IP address obtained by external packet monitoring 
software.
