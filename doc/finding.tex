\section{Finding the syscall table at runtime}
Using predefined adresses limits the rootkit to the machine it was 
built on. This of course is rarely useful, so this rootkit comes with a 
mechanism to find the syscall table at runtime. \par
The interrupt \texttt{int 0x80} for performing system calls was replaced by 
\texttt{SYSCALL} and \texttt{SYSENTER} for 64-bit system calls. These processor 
functions don't directly jump into a function in the \texttt{IDT} but call a 
handler provided by the OS. The handler's address is located in the 
\texttt{LSTAR} machine specific register. To get to the the syscall table 
nevertheless, we get the address of the system call handler using the 
\texttt{rdmsrl} function (read 
machine specific register long, indicating a 64-bit address). The source code 
of the handler function is found at \texttt{arch/x86/kernel/entry.S:629}. It 
calls the system calls with the following instruction: 
\verb+call *syscall_table(,%rax,8)+. Using an opcode table, we were able to 
assemble this instruction to the sequence \texttt{0xFF 0x14 0xC5}. The following 
32-bit address contains the lower part of the address of the syscall table,
which then is or-ed bitwise with 0xffffffff00000000, because the RIP 
relative addressing omits the first 32 bits of the address. This address now 
contains the full 64 bit address of the syscall table and is ready to use.