\section{File Hiding}
Using the tool strace on the command \texttt{ls}, we have found out that for file-listing the system call 
\begin{center}
\lstset{escapechar=,style=customc}
\begin{lstlisting}
long sys_getdents(unsigned int fd, struct linux_dirent __user *dirent, unsigned 
int count);
\end{lstlisting}
\end{center}
is used to get directory entries. This function will read a set of data of the \verb+linux_dirent+ structure (see figure \ref{ls:dirent}) from the directory referenced by the file descriptor \texttt{fd} into the buffer, which is pointed to \texttt{dirent}. The size of the buffer is specified by the parameter \texttt{count}. If no error occurs, the number of read bytes will be returned.\\
To hide some specific files (e.g. file name is is prefixed with ``rootkit\_'') when listing directories, we have replaced this syscall function with our own function \verb+my\_getdents+ (see \verb+file\_hiding.c+) using the mechanism introduced in section \ref{sec:syscallHooking}. In the function \verb+my\_getdents+, first the original \verb+sys\_getdents+ is called to fill the buffer. Then to be hidden \verb+linux\_dirent+ data is overwritten by moving the rest of \verb+linux\_dirent+ data forward. The returned value is reduced by the appropriate multiple of the length of the structure \verb+linux\_dirent+.\\ 
In order to look up files with the specific name, the data of the structure \verb+linux\_dirent+ is iterated using the return value of the original \verb+sys\_getdents+ function and the \verb+d\_reclen+ value of each \verb+linux\_dirent+ data. The name of files is stored in the \verb+d\_name+ field. 
\begin{center}
\begin{figure}[h]
\lstset{escapechar=,style=customc}
\begin{lstlisting}
struct linux_dirent {
        unsigned long   d_ino;		 	/* Inode number */
        unsigned long   d_off;  		/* Offset to next linux_dirent */
        unsigned short  d_reclen;		/* Length of this linux_dirent */
        char            d_name[];		/* Filename (null-terminated) */
};
\end{lstlisting}
\caption{The \texttt{linux\_dirent} structure defined in \texttt{/fs/readdir.c}}
\label{ls:dirent}
\end{figure}
\end{center}
This feature is implemented in \verb+file\_hiding.c+ and \verb+file\_hiding.h+. If the function \verb+hide\_files+ is called, the files with the prefix ``rootkit\_'' will not be listed, when the command \texttt{ls} is used. However they still are accessible, if you know the exact file name. After calling the function \verb+unhide\_files+, the program ``ls'' can list all existing files again. 
