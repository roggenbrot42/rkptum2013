\section{File Hiding}
Using the tool strace on the command \texttt{ls}, you can find out that for 
file-listing the system call 
\begin{center}
\lstset{escapechar=,style=customc}
\begin{lstlisting}
long sys_getdents(unsigned int fd, struct linux_dirent __user *dirent, unsigned 
int count);
\end{lstlisting}
\end{center}
are used to get directory entries. This function will read a set of data of 
\verb+linux_dirent+ structure (see figure \ref{ls:dirent}) from the directory 
relative to the file descriptor \texttt{fd} into the buffer, which is pointed 
by \texttt{dirent}. The size of the buffer is specified by the parameter 
\texttt{count}. If no error occurs, the number of read bytes will be returned.\\
To hide some specific files (e.g. file name is with the prefix ``rootkit\_'') 
when listing directories, we have replace this syscall function with our own 
function \texttt{my\_getdents} (see \texttt{file\_hiding.c}) using the mechanism 
introduced in section \ref{sec:syscallHooking}. In the function 
\texttt{my\_getdents}, the original \texttt{sys\_getdents} is first at all 
called to fill the buffer. Then to hidden \texttt{linux\_dirent} data are 
override by moving the rest of \texttt{linux\_dirent} data forwards and the 
returned value is reduced of appropriate time of the length of the 
\texttt{linux\_dirent} length.\\ 
In order to look up files with the specific name, the struct 
\texttt{linux\_dirent} data are iterated through using the return value of the 
original \texttt{sys\_getdents} function and the $d\_reclen$ value of each 
\texttt{linux\_dirent} data. The name of files is stored in the \texttt{d\_name} 
field. 
\begin{center}
\begin{figure}
\lstset{escapechar=,style=customc}
\begin{lstlisting}
struct linux_dirent {
        unsigned long   d_ino;
        unsigned long   d_off;
        unsigned short  d_reclen;
        char            d_name[];
};
\end{lstlisting}
\caption{The \texttt{linux\_dirent} structure defined in \texttt{/fs/readdir.c}}
\label{ls:dirent}
\end{figure}
\end{center}
This feature is implemented in \texttt{file\_hiding.c} and \texttt{file\_hiding.h}. If the 
function \texttt{hide\_files} is called, the files with the prefix ``rootkit\_'' will 
not be listed by command \texttt{ls}. However they are still accessible, if you know 
the exact file name. After calling the function \texttt{unhide\_files}, the program 
``ls'' can list all existing files again. 
