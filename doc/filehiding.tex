\section{File Hiding}
Using the tool strace on the command $ls$, we have found out that for file-listing the system call 
\begin{center}
\lstset{escapechar=,style=customc}
\begin{lstlisting}
long sys_getdents(unsigned int fd, struct linux_dirent __user *dirent, unsigned int count);
\end{lstlisting}
\end{center}
is used to get directory entries. This function will read a set of data of the $linux\_dirent$ structure (see figure \ref{ls:dirent}) from the directory referenced by the file descriptor $fd$ into the buffer, which is pointed to $dirent$. The size of the buffer is specified by the parameter $count$. If no error occurs, the number of read bytes will be returned.\\
To hide some specific files (e.g. file name is is prefixed with ``rootkit\_'') when listing directories, we have replaced this syscall function with our own function $my\_getdents$ (see $file\_hiding.c$) using the mechanism introduced in section \ref{sec:syscallHooking}. In the function $my\_getdents$, first the original $sys\_getdents$ is called to fill the buffer. Then to be hidden $linux\_dirent$ data is overwritten by moving the rest of $linux\_dirent$ data forward. The returned value is reduced by the appropriate multiple of the length of the structure $linux\_dirent$.\\ 
In order to look up files with the specific name, the data of the structure $linux\_dirent$ is iterated using the return value of the original $sys\_getdents$ function and the $d\_reclen$ value of each $linux\_dirent$ data. The name of files is stored in the $d\_name$ field. 
\begin{center}
\begin{figure}
\lstset{escapechar=,style=customc}
\begin{lstlisting}
struct linux_dirent {
        unsigned long   d_ino;		 	/* Inode number */
        unsigned long   d_off;  		/* Offset to next linux_dirent */
        unsigned short  d_reclen;		/* Length of this linux_dirent */
        char            d_name[];		/* Filename (null-terminated) */
};
\end{lstlisting}
\caption{The $linux\_dirent$ structure defined in $/fs/readdir.c$}
\label{ls:dirent}
\end{figure}
\end{center}
This feature is implemented in $file\_hiding.c$ and $file\_hiding.h$. If the function $hide\_files$ is called, the files with the prefix ``rootkit\_'' will not be listed, when the command $ls$ is used. However they still are accessible, if you know the exact file name. After calling the function $unhide\_files$, the program ``ls'' can list all existing files again. 
