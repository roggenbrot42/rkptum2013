\section{Module Hiding}
There are two different ways to obtain information about the currently loaded 
kernel modules: \texttt{/proc/modules} and \texttt{/sys/module/}. To hide from 
these two sources, two different methods are required.
\subsection{Hiding from /proc/modules}
Hiding from \texttt{/proc/modules} is very crucial since \texttt{lsmod} gets 
its information from this file. Because the proc files are not dynamically 
generated, direct changes to the file are pointless. Instead we found it to be 
very efficient to remove the module from the kernel's internal module list 
using the following code:
\begin{lstlisting}
 tmp_head = THIS_MODULE->list.prev;
 list_del(&THIS_MODULE->list);	
\end{lstlisting}
By saving the pointer to the previous item in the list, we ensure that we can 
reattach the \texttt{module} structure to the list.

\subsection{Hiding from /sys/module/}
The \texttt{/sys/module/} directory contains a folder for each loaded module. 
These folders get their information through the \texttt{kobject} structure, a 
member of \texttt{struct module\_kobject}, which in return is a member of 
\texttt{struct module}. Deleting this \texttt{kobject} struct makes the module 
vanish from \texttt{/sys/module/} but leads to a messed up reference counter 
and thus, prevents the kernel from unloading the kernel.\\
Instead, we chose to manipulate the file operations of the \texttt{/sys/} 
inode. The inode structure be can easily obtained from the \texttt{dentry} 
structure returned by \texttt{kern\_path()}. Using this inode, we 
replaced the \texttt{readdir} operation with our own method, which in turn 
makes use of 
our custom \texttt{my\_filldir\_t} function. This custom 
\texttt{my\_filldir\_t} function simply returns 0 if 
the filename is equal to that of our rootkit or else calls the original 
method.\newline
\hfill\\
The implementation of the module hiding can be found in \verb+code_hiding.h+ and \verb+code_hiding.c+.
\\Because the rootkit is so well-hidden, it is impossible to remove the module 
without unhiding it first. The unhiding mechanism simply performs a rollback to 
the old status and can be initiated using the command and control channel. 
See section \ref{cnc} for more information.
