\section{System Call Hooking}
\label{sec:syscallHooking}
A system call is a manner to request a service from the kernel in the user 
space. It provides an essential interface between user space processes and the 
operating system.\\
In Linux, all the syscall functions are declared in \verb+linux/syscalls.h+. The 
address to these functions are store in an array called \verb+sys_call_table+. 
A simple way to hook the provided services from the kernel is to replace the 
appropriate location in \verb+sys_call_table+. The address of the 
\verb+sys_call_table+ can be found in the system map file 
(\verb+/boot/System.map-\$(uname -r)+), which is a symbol table used by the 
Linux kernel. But the type of the symbol \verb+sys_call_table+ is read-only data 
(R). That means, we should remove the write protection mechanism before system 
call hooking.\\
On x86-64 microprocessors, there are a series of control register to control the 
general CPU behavior. The \texttt{CR0} register stores 64 bits control flags for 
basic operations. The bit 16 (WP) determines, whether the CPU can write to 
read-only marked data. Hence, in order to disable the write protection, we need 
just write 1 to the 16th bit in the CR0 register using the kernel function 
\verb+read_cr0+ and \verb+write_cr0+ (defined in \verb+asm/paravirt.h+). The 
implementation is in \texttt{hooking.h}.\\
To see which system calls a program used, you can use a debugging tool called 
strace. Then you can try to change the kernel service by hooking any syscall 
functions with the mechanism introduced above. You need store the origin syscall 
function pointer, so that it is possible to recovery the kernel service by 
unloading your rootkit module. 

