\section{Command and Control}\label{cnc}
The command and control code (implemented in \verb+command.h+ and \verb+command.c+) 
parses all input from stdin for possible user 
commands. To do so, it hooks the \texttt{read} system call and filters for the 
stdin file descriptor. Because many users can write to this file descriptor at 
the same time. We created a list to collect \texttt{stdin} data for specific tasks.
As a data structure for this list, we defined the following struct:
\lstset{escapechar=,style=customc}
\begin{lstlisting}
struct taskinput_buffer{
	char buf[INPUTBUFLEN];
	unsigned short bufpos;
	char * name;
	struct list_head list;
};
\end{lstlisting}
We decided to identify these input buffers by the name of the first 
file in the list of open file descriptors of the current process. 
This could be simplified by just using the process' ID.
Tests have shown that the autocompletion used by bash causes filename to be 
\texttt{NULL}, which is why autocompleted commands do not work. This could be 
avoided by using process IDs instead of filenames.\\
The \verb+taskinput_buffer+ structure contains the current location 
in the buffer, so the program can react appropriately if the user hits 
backspace or presses enter. Pressing backspace decrements the location pointer 
while pressing enter causes the rootkit to parse the buffer for commands and 
resets the location pointer.\\
To actually parse the commands, the rootkit holds a list of all known commands. 
This list comprises of instances of the following structure:
\lstset{escapechar=,style=customc}
\begin{lstlisting}
struct command{
	char name[CMDLEN];
	size_t namelen;
	u64 hashcode;
	enum arg_t arg_type;
	void * handler;
	struct list_head list;
}
\end{lstlisting}
The 64 bit \texttt{hashcode} member stores up to eight characters of 
each command and is used as a unique identifier. This saves time when traversing 
the list of commands, because it is potentially faster to compare integers than 
strings. Commands can have three types of arguments: none, integer and list of 
integers (comma separated).
Furthermore, each command has a pointer to a \texttt{handler} function that 
will execute the command. In the case of integer lists, the handler function is 
called for each item separately. The implemented user commands, which are initialized when loading the module, as well as needed parameters are listed in table \ref{tab:com}
\begin{table}[h]
\centering
\begin{tabular}{|l|l|l|}
\hline
Command & Parameter & Description \\\hline\hline
hideme & none & Hide module \\\hline
unhideme & none & Show module \\\hline
hidepid & pid1,pid2,... & Hide processes \\\hline
hidefile & none & Hide all files with prefix "rootkit\_"\\\hline
unhidef & none & Show files \\\hline
socktcp & port1,port2,... & Hide TCP sockets \\\hline
sockudp & port1,port2,... & Hide UDP sockets \\\hline
sueme & none & Grant current user root access\\\hline
\end{tabular}
\caption{List of all commands known to the rootkit}
\label{tab:com}
\end{table}
