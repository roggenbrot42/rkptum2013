\section{Process Hiding}
The tool \textit{ps} gets its information from the process filesystem. The 
\texttt{/proc/} directory contains a folder for every process 
currently run on the system. To hide certain processes from \textit{ps} we chose 
to simply hide the corresponding folders in \texttt{/proc/}. 
The PIDs can be passed to the module as a parameter or at runtime via command 
(see \ref{cnc}). The hiding procedure is fairly easy and reused at many points 
in the rootkit.

The idea is to get the \texttt{/proc/} root inode structure and replace its 
\texttt{readdir} operation by the rootkit's own
\texttt{proc\_readdir}. This function simply replaces the provided 
\texttt{filldir} function by a filldir function that filters out the 
directories of all given PIDs. 