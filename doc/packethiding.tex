\section{Packet Hiding}
%- Libcap sniffs network traffic through a PF_PACKET socket and receive clones 
%      of packets. To hide packets, sniffed by libcap, one needs to hook the 
%      function for receiving packets of packet sockets, used by libcap. The 
%      functions that deal with the packets are implemented in 
%      net/packet.af_packet.c: 
%        int tpacket_rcv(struct sk_buff *, struct net_device *dev, struct packet_type *, struct net_device *); 
%    - There are two ways to hook the function tpacket_rcv:
%      - Change the operation pointer (point to the function tpacket_rcv) of the 
%        struct packet_sock by hooking the syscall funktion:
%        long sys_socket(int, int, int);
%        (Not easy to reverse the pointer when unloading, especially to deal with 
%        more than one packet socket.)
%      - Inject assembler code to the origin tpacket_rcv function. One can use
%        the assembler code:
%          pushq imm32;
%          retq;
%        to cause the instruction pointer to jump to a hooking function.
%        (Leads to exception, caused by too frequent traffic, because the original
%        function is continously changed to the modified code version and reversed 
%        to hide the packets of given IP address and report other packets at the
%        same time)
%----------------------------------------------------------------------------------
%Implementation Details
%    - void hide_packets(void): 
%      - Read the address of the function packet_rcv using the kernel function 
%        kallsys_lookup_name("tpacket_rcv"). 
%      - Store the original 6 byte code of the function packet_rcv and replace it
%        with the assembler code.
%    - void unhide_packets(void): reverse the code injection.
%-----------------------------------------------------------------------------------
%Additional Notes
%	  Please load the module, when tcpdump runs locally or via ssh with filter to not 
%    sniff this ssh traffic. 
%    Loading the module leads to exception, when tcpdump is runned via ssh without 
%    filtering, because running tcpdump by ssh leads to loops and hence destroys 
%    the hiding mechanism (See Technical note above).
