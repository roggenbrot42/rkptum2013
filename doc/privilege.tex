\section{Privilege escalation}
Every \texttt{struct task\_struct} in Linux has two sets of credentials. 
\texttt{real\_cred} points to the real task credentials and \texttt{cred} points 
to the effective task credentials. For privilege escalation, it is sufficient to 
alter the effective credentials.\\
To manipulate these credentials it is required to use the appropriate functions 
\texttt{prepare\_creds()} and \texttt{commit\_creds()}.
To grant the current user root access, we fetch the current task's 
\texttt{task\_struct} using the \texttt{current} macro. Them, the rootkit sets 
the real, effective and file system operation user IDs and group IDs to zero 
and commits the changes. Because the privilege escalation must be requested by 
the current user using the ``sueme'' command, the current task always is a 
virtual terminal. Empowering this process means that all subsequent processes 
created by the user have root credentials.
